\documentclass[times, 12pt]{article}
\usepackage{times}

\usepackage{amsmath, amsfonts, amssymb}                 % Packages to allow inclusion of graphics
\usepackage{graphics}                 % Packages to allow inclusion of graphics
\usepackage{hyperref}                 % For creating hyperlinks in cross references
\usepackage{listings}                 % Source code listing

    \addtolength{\oddsidemargin}{-.875in}
    \addtolength{\evensidemargin}{-.875in}
    \addtolength{\textwidth}{1.75in}

%    \addtolength{\Tmargin}{-.875in}
    \addtolength{\textheight}{1.75in}

\title{Advanced Linear Algebra Solutions}
\author{Matthew Bennett $\diamondsuit$ Printed \today }
 \date{ }

\begin{document}
\pagestyle{plain} \maketitle

\section{Definitions}
\begin{description}
\item[Non-Singular] Square matrix having a unique inverse.
\item[Singular] Square matrix having no inverse.
\item[Basis of $V$] A Linearly Independent spanning set for vector
space $V$.

\end{description}


\section{Friday Jun 02, 2006}

\begin{description}
\item[4.1.1] Determine which subsets of $\mathfrak{R^n}$ are
subspaces.
\begin{enumerate}
\renewcommand{\labelenumi}{(\alph{enumi})}
\item $\{\textbf{x} | x_i \geq 0\}$ No. It is not closed under scalar
multiplication by a negative number.
\item $\{\textbf{x} | x_1 = 0\}$] Yes.
\item $\{\textbf{x} | x_1 x_2 = 0\}$ No. Let $a_1 = 0, b_1 \neq
0$. Then $\textbf{a} + \textbf{b} \notin \{\textbf{x} | x_1 x_2 =
0\}$
\item $\{\textbf{x} | \displaystyle\sum_{i=1}^nx_i = 0   \}$] Yes.
\item $\{\textbf{x} | \displaystyle\sum_{i=1}^nx_i = 0  \}$]
No. It is not closed under scalar multiplication.
\item $\{\textbf{x} | \textbf{Ax} = \textbf{b}$, and $\textbf{A}_{m\times n} \neq \textbf{0}$ and $\textbf{b}_{m\times 1} \neq \textbf{0} \}$]
\\ No (Why? Also, What is this Notation?)
\\ $\bullet$ Unresolved.
\end{enumerate}
\end{description}

\begin{description}
\item[4.1.2] Determine which subsets of $\mathfrak{R^{n\times n}}$ are
subspaces.
\begin{enumerate}
\renewcommand{\labelenumi}{(\alph{enumi})}
\item The Symmetric Matrices? \\ Yes.
\item The Diagonal Matrices? \\ Yes.
\item The Nonsingular Matrices? \\ No, because it is not closed
under addition.
\item The Singular Matrices? \\ No, not closed under
addition.
\item The Triangular Matrices? \\ No, because $\textbf{U} +
\textbf{L}$ is not triangular.
\item The Upper-Triangular Matrices? \\ Yes.
\item All Commuting Matrices of \textbf{A}? \\ Yes.
\item All matrices \textbf{A} such that $\textbf{A}^2 = \textbf{A}$
\\ No, because it is not closed under scalar multiplication.
(However, because of the pythagorean theorem it is closed under
addition.)
\item All matrices \textbf{A} such that $trace(\textbf{A}) = 0$ \\
Yes. (Proof?) If that is true, then
$$\displaystyle\sum_{i=1}^nx_{ii} = 0$$ Therefore, it is closed under
addition. Also, scalar multiplication, since it is also true that
$$\displaystyle\sum_{i=1}^n\alpha x_{ii} = \displaystyle\alpha\sum_{i=1}^nx_{ii}
= 0$$
\end{enumerate}
\end{description}

\begin{description}
\item[4.1.3] If vector space $\mathcal{X}$ is a plane passing through the origin in $\mathfrak{R}^3$ and $\mathcal{Y}$ is the line through the origin that is perpendicular
to $\mathcal{X}$, then what is $\mathcal{X + Y}$? \\[1cm]
Answer: $\mathfrak{R}^3$
\end{description}

\begin{description}
\item[4.1.4] Why must a real or complex nonzero vector field contain an infinite number of vectors? \\[1cm]
Answer: A vector space is defined as all scalar multiples of any
vectors contained, else it would not be closed under scalar
multiplication. The only counterexample is the trivial vector space
defined on $\emptyset$.
\end{description}

\begin{description}
\item[4.1.5] Describe the subspace defined by the given column-vector matrices\\[1cm]
$$\left(
  \begin{array}{ccc}
    \left(
       \begin{array}{c}
         1 \\
         3 \\
         2 \\
       \end{array}
     \right)
     & \left(
          \begin{array}{c}
            2 \\
            6 \\
            4 \\
          \end{array}
        \right)
      & \left(
          \begin{array}{c}
            -3 \\
            -9 \\
            -6 \\
          \end{array}
        \right)
       \\
  \end{array}
\right)$$ Column 3 is an elementary transformation of column 1.
Therefore, the subspace defined should be a line, since only two
column vectors are linearly independent.
$$\left(
  \begin{array}{ccc}
    \left(
       \begin{array}{c}
         -4 \\
         0 \\
         0 \\
       \end{array}
     \right)
     & \left(
          \begin{array}{c}
            0 \\
            5 \\
            0 \\
          \end{array}
        \right)
      & \left(
          \begin{array}{c}
            1 \\
            1 \\
            0 \\
          \end{array}
        \right)
       \\
  \end{array}\right)$$
These vectors are all linearly independant. The space should be a
plane, since the third component has a coefficient of zero in all
vectors.
$$\left(
  \begin{array}{ccc}
    \left(
       \begin{array}{c}
         1 \\
         0 \\
         0 \\
       \end{array}
     \right)
     & \left(
          \begin{array}{c}
            1 \\
            1 \\
            0 \\
          \end{array}
        \right)
      & \left(
          \begin{array}{c}
            1 \\
            1 \\
            1 \\
          \end{array}
        \right)
       \\
  \end{array}\right)$$
These vectors are both independent and span all of $\mathfrak{R}^3$
\end{description}

\section{Wed June 7, 2006}
\begin{description}
\item[4.2.1] Determine spanning sets for
$$A = \left(
    \begin{array}{ccccc}
      1 & 2 & 1 & 1 & 5 \\
      2 & -4 & 0 & 4 & -2 \\
      1 & 2 & 2 & 4 & 9 \\
    \end{array}
  \right)
 $$

 Answer: Gauss Elimination yields a rank-2 matrix
$$\left(
    \begin{array}{ccccc}
      1 & 2 & 1 & 1 & 5 \\
      0 & 0 & 2 & 6 & 8 \\
      0 & 0 & 0 & 0 & 0 \\
    \end{array}
  \right)
 $$

$$R(\textbf{A}) = span\left(
                        \begin{array}{cc}
                          \left(
                            \begin{array}{c}
                              1 \\
                              -2 \\
                              1 \\
                            \end{array}
                          \right)
                           , & \left(
                               \begin{array}{c}
                                 1 \\
                                 0 \\
                                 2 \\
                               \end{array}
                             \right)
                            \\
                        \end{array}
                      \right) $$
Since the range space of \textbf{A} is the same as the space spanned
by the columns of \textbf{A}. Similarly,

$$R(\textbf{A}^T) = span\left(
                        \begin{array}{cc}
                          \left(
                            \begin{array}{c}
                              1 \\
                              2 \\
                              1 \\
                              1 \\
                              5 \\
                            \end{array}
                          \right)
                           , & \left(
                               \begin{array}{c}
                                 0 \\
                                 0 \\
                                 2 \\
                                 6 \\
                                 8 \\
                               \end{array}
                             \right)
                            \\
                        \end{array}
                      \right) $$

$$N(\textbf{A}) = \left(
                    \begin{array}{ccc}
                      \left(
                        \begin{array}{c}
                          -2 \\
                          1 \\
                          0 \\
                          0 \\
                          0 \\
                        \end{array}
                      \right)
                       & \left(
                           \begin{array}{c}
                             2 \\
                             0 \\
                             -3 \\
                             1 \\
                             0 \\
                           \end{array}
                         \right)
                        & \left(
                            \begin{array}{c}
                              -1 \\
                              0 \\
                              -4 \\
                              0 \\
                              1 \\
                            \end{array}
                          \right)
                         \\
                    \end{array}
                  \right)
$$
$$N(\textbf{A}^T) = $$
\end{description}

\end{document}
